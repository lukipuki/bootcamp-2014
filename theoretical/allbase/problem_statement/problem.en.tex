\problemname{All Your Base Are Belong to Palindromes}
Most of the time, humans have 10 fingers. This fact is the main reason that our numbering system is
base-10: the number 257 really means $2 \times 10^2 + 5 \times 10^1 + 7 \times 10^0$. Notice that
each digit in base-10 is in the range from $0, \dots,  9$.

Of course, there are other bases we can use: binary (base-2), octal (base-8) and hexadecimal
(base-16) are common bases that really cool people use when trying to impress others. In base-$b$,
the digits are in the range from $0, \dots, b - 1$, with each digit (when read from right to left)
being the multiplier of the next larger power of $b$.

So, for example 9 (in base-10) is:
\begin{itemize}
  \item 9 in base-16
  \item 11 in base-8 ($1 \times 8^1 + 1 \times 8^0 = 9$)
  \item 1001 in base-2 ($1 \times 2^3 + 0 \times 2^2 + 0 \times 2^1 + 1 \times 2^0 = 9$)
\end{itemize}

Noticing the above, you can see that 9 is a palindrome in these 3 different bases. A palindrome is
a sequence which is the same even if it was written in reverse order: English words such as “dad”,
“mom” and “racecar” are palindromes, and numbers like 9, 11, 1001 are also palindromes.

Given a particular number $X$ (in base-10), for what bases $b$ $(2 \le b < X)$ is the representation
of $X$ in base-$b$ a palindrome?

\section*{Input}
There will be one line, containing the integer $X$ $(2 \le X \le 1\,000\,000\,000\,000)$.
For partial credit, you may assume $X \le 100\,000$.

\section*{Output}
The output should consist of a sequence of increasing integers, each on its own line, indicating
which bases have the property that $X$ written in that base is a palindrome. Note that we will only
concern ourselves with bases which are less than $X$, and that the first possible valid base is 2.
